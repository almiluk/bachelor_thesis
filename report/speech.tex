\documentclass[14pt]{extarticle}
\usepackage{cmap}
\usepackage[utf8]{inputenc}
\usepackage[T2A]{fontenc}
\usepackage[english, russian]{babel}

\usepackage{geometry}
\geometry{
	a4paper,	% size of paper
	left=10mm,	%
	right=15mm,	% margins
	top=10mm,	%
	bottom=10mm,%
}

\usepackage{setspace}
\singlespacing % line spacing

\usepackage{ragged2e}
\hyphenchar\font=-1
\sloppy

\begin{document}

\begin{enumerate}
	\item Автоматизация процесса разработки и отладки программ для микроконтроллеров и для созданных на их основе устройств интернета вещей является важной и острой задачей для разработчиков.
	Это вызвано, в первую очередь, ограниченностью интерфейсов микроконтроллеров для взаимодействия с пользователем и отсутствием высокоуровневых средств отладки.
	\item Особо остро эти проблемы встают перед платформой визуальной аналитики и научной визуализации SciVi, разработанной сотрудниками кафедры МОВС.
	Это обусловлено тем, что в данной платформе микроконтроллеры используются не просто для исполнения единожды загруженных в них программ, а для для интерпретации пользовательских алгоритмов, передающихся на микроконтроллеры в процессе работы.
	Причём передаются алгоритмы в виде онтологий, сжатых с помощью набора программных средств EON, который так же разработан сотрудниками кафедры.
	Решению проблем автоматизации программирования микроконтроллеров и посвящена моя работа.
	\item
	Объект исследования:  автоматизация периферийных вычислений.
	Предмет исследований: средства платформы SciVi для организации онтологически-управляемых периферийных вычислений.
	\item Цель ВКР, в рамках котрой была проведена НИР -- разработка требуемых платформой SciVi программных средств для автоматизации программирования микроконтроллеров.
	\item А цель самой НИР: Разработать программный модуль с высокоуровневым интерфейсом, позволяющий удобно и без необходимости ручной настройки сохранять и считывать информацию из энергонезависимой памяти микроконтроллеров, в соответствии с требованиями платформы SciVi.
	Платформе SciVi такой модуль необходим для повышения уровня автономности и отказоустойчивости используемых устройств интернета вещей за счёт автоматизации восстановления состояния этих устройств после их перезагрузки.
	Отмечу, что в микроконтроллерах обычно используются электрически стираемые перепрограммируемые постоянные запоминающие устройства (Electrically Erasable Programmable Read-Only Memory, EEPROM), позволяющие стереть сохранённые данные, а затем записать новые.
	\item Для достижения цели НИР были поставлены следующие задачи:
	\begin{enumerate}
		\item Составить требования к необходимому программному модулю.
		\item Исследовать существующие средства для работы с энергонезависимой памятью и, в частности, EEPROM микроконтроллеров.
		\item При возможности, выбрать одно из таких средств для использования в качестве основы разрабатываемого модуля.
		\item Разработать программный модуль для работы с EEPROM микроконтроллеров, соответствующий всем поставленным требованиям.
		\item Провести тестирование и отладку разработанного программного модуля.
		\item Интегрировать разработанный модуль в платформу SciVi.
	\end{enumerate}
	\item Перед необходимым платформе SciVi программным модулем были поставлены следующие требования:
	\begin{enumerate}
		\item Наличие возможности сохранять и считывать данные произвольной структуры из EEPROM.
		Это основное назначение модуля.
		\item Для достижения необходимого уровня удобства в использовании, обращение к данным в EEPROM должно производиться по некоторым, заданным пользователем, идентификаторам, без необходимости ручных манипуляций с EEPROM адресами.
		\item Минимизация количества операций записи в EEPROM.
		Общее количество циклов перезаписи для EEPROM ограничено и обычно составляет от 100,000 до 1,000,000 циклов, поэтому количество операций записи необходимо сводить к минимуму.
		\item  Программный модуль должен выполняться на микроконтроллерах серии ESP8266, так как именно их использует SciVi и, по возможности, на платформе Arduino, так как она пользуется наибольшей популярностью в мире.
	\end{enumerate}
	\item Микроконтроллеры серии ESP8266 имеют важную особенность: в ESP8266 в качестве EEPROM используется flash-память, позволяющая стирать данные только большими блоками.
	В то время, как во многих других микроконтроллерах EEPROM позволяет стирать отдельные байты, не уменьшая ресурс остальных.
	Причём в ходе проведённой работы при анализе исходного кода библиотек для ESP8266 было установлено, что размер стираемого блока данных равен всему объёму памяти, доступному пользователю.
	Т.е. для обновления даже одного байта в ESP8266 необходимо, фактически, перезаписать в EEPROM все данные.
	\item На первом этапе работы были проанализированы наиболее популярные, из существующих, решения для управления энергонезависимой памятью микроконтроллеров.
	Первое из них -- стандартная библиотека, входящая в состав набора инструментов для программирования микроконтроллеров Arduino IDE.
	Она предоставляет низкоуровневые функции для чтения и записи в EEPROM и обёртку вокруг функции записи, производящую перезапись только в случае, если записываемы данные не совпадают с уже находящимися в EEPROM.
	Однако все эти функции требуют ручного указания адреса для записи или чтения, что, очевидно, не соответствует поставленным требованиям.
	\item Второе рассмотренное решение - библиотека EEManager.
	Данная библиотека включает в себя механизм отложенной записи, не позволяющий производить перезапись данных в EEPROM слишком часто.
	И механизм ключа первой записи, позволяющий упростить процесс записи начального значения в EEPROM и последующего чтения из него.
	Но при этом данная библиотека также требует явно указывать адреса EEPROM для взаимодействия, что так же нарушает поставленные требования.
	\item Последняя из рассмотренных библиотек -- EEPROMWearLevel.
	Данная библиотека позволяет обращаться к блокам данных в EEPROM по их индексам, а не адресам.
	Это также не позволяет использовать библиотеку в нескольких независимых программных модулях из-за того, что множество индексов едино для всего EEPROM и не может быть разделено между модулями, однако EEPROMWearLevel подходит к этой цели ближе всех других рассмотренных библиотек.
	Кроме того, библиотека содержит ряд механизмов, способных значительно уменьшить износ EEPROM с побайтовой перезаписью.
	Однако, при использовании с flash памятью эти механизмы наоборот только ускорят износ, кроме того, их реализация содержит платформозависимый код, выполнение которого на микроконтроллерах ESP8266 невозможно.
	\item На основе проведённого анализа был сделан вывод о необходимости создания собственной библиотеки для работы с EEPROM. На первых этапах проектирования было решено в качестве идентификаторов блоков данных использовать текстовые имена.
	А для уменьшения вероятности коллизий этих имён, разделить все данные, хранящиеся в EEPROM на разделы -- по сути, пространства имён для блоков данных.
	\item С учётом этого была составлена следующая диаграммы прецедентов использования разрабатываемой библиотеки, на основе которой производилось дальнейшее проектирование библиотеки.
	\item Затем была разработана структура хранения данных в EEPROM.
	В ней стоит отметить наличие системного раздела памяти, которых хранит информацию об остальных разделах и о EEPROM в целом.
	Остальные же разделы описываются в EEPROM так же, как любые пользовательские данные -- за счёт их помещения в EEPROM-переменные.
	\item Наконец, была составлена диаграммы классов разрабатываемом библиотеки, включающая в себя классы для сущностей EEPROM-переменной, раздела EEPROM и описания EEPROM в целом.
	Первая версия библиотеки была реализована в соответствии с разработанными моделями.
	Однако при её тестировании было явлено, что несмотря на то, что её логика работает в соответствии с ожиданиями, её интерфейс является не достаточно высокоуровневым для достижения необходимого уровня удобства разработки.
	\item В качестве последнего, достигнутого на данный момент результата, были составлены общие требования по переработке и улучшению библиотеки.


\end{enumerate}

\end{document}
