\chapter*{ПРИЛОЖЕНИЕ А} \label{app:appendix-eeprom-usage-example}

\begin{center}
    \bfseries Пример использования разработанной библиотеки менеджера EEPROM
\end{center}

\begin{verbatim}
#include <EEPROMVariable.h>


void setup() {
    Serial.begin(115200);

    EEPROMVar<uint16_t> boot_cnt("boot_cnt", 1);

    Serial.println(boot_cnt);

    boot_cnt = boot_cnt + 1;
}

void loop() {

}
\end{verbatim}


\chapter*{ПРИЛОЖЕНИЕ Б} \label{app:appendix-ssdp-usage-example}

\begin{center}
    \bfseries Пример использования разработанной реализации протокола SSDP
\end{center}

\begin{verbatim}
#include <ESP8266WebServer.h>
#include <ESP8266WiFi.h>
#define NO_GLOBAL_SSDP
#include <almilukESP8266SSDP.h>

#define NETNAME "SSID"
#define PASSWORD "PASSWORD"

class mySSDPClass : public SSDPClass {
public:
	int response_cnt = 0;

	mySSDPClass() {
		setManufacturer("almiluk");
		setManufacturerURL("https://github.com/almiluk");
		setModelName("almilukESP8266SSDP_test");
		setDeviceType("almiluk-domain", "esp8266-ssdp-test", "1.0");
		SSDPServiceType services[] = {
			{"almiluk-domain", "service1", "v1"},
			{"some-other-domain", "service1", "1.1.0"},
			{"some-other-domain", "service2", "abcd"}
		};
		setServiceTypes(services, 3);
		// You can set all combinations of attributes you want.
	}

protected:

	void on_response() {
		Serial.print("Sending SSDP response for target: ");
		Serial.println(getAdvertisementTarget());
		addHeader("resp_header", "resp_header_val");
		response_cnt++;
	}

	void on_notify_alive() {
		Serial.print("Sending SSDP alive notification for target: ");
		Serial.println(getAdvertisementTarget());
		addHeader("alive_header", "alive_header_val");
	}

	void on_notify_bb() {
		Serial.print("Sending SSDP byebye notification for target: ");
		Serial.println(getAdvertisementTarget());
		addHeader("byebye_header", "byebye_header_val");
	}
};

mySSDPClass g_ssdp;
ESP8266WebServer g_webServer(80);

void setup() {
	Serial.begin(115200);

	WiFi.mode(WIFI_STA);
	WiFi.begin(NETNAME, PASSWORD);
	Serial.print("Waiting for WiFi connection");
	while (!WiFi.localIP().isSet()) {
		Serial.print('.');
		delay(500);
	}
	Serial.println("\nConnected to WiFi");

	if (g_ssdp.begin())
		Serial.println("SSDP begun");
	else
		Serial.println("SSDP init failed");

	g_webServer.on("/ssdp/schema.xml", []() {
		g_ssdp.schema(g_webServer.client());
		});
	g_webServer.begin();

	Serial.println("Web server started");
}

void loop() {
	if (g_ssdp.response_cnt == 9) {
		delay(10000);
		g_ssdp.end();
		g_ssdp.response_cnt++;
	} 
	g_ssdp.loop();
	g_webServer.handleClient();
	delay(16);
}
\end{verbatim}


\chapter*{ПРИЛОЖЕНИЕ В} \label{app:appendix-search-func}

\begin{center}
    \bfseries Функция поиска устройств и сервисов
\end{center}

\begin{verbatim}
def get_ssdp_list(st_val: str = "ssdp:all", timeout: int = 3):
    SSDP_ADDR = "239.255.255.250";
    SSDP_PORT = 1900;
    SSDP_MX = timeout;

    ssdpRequest = ("M-SEARCH * HTTP/1.1\r\n"
                    + "HOST: %s:%d\r\n" % (SSDP_ADDR, SSDP_PORT)
                    + "MAN: \"ssdp:discover\"\r\n"
                    + "MX: %d\r\n" % (SSDP_MX, )
                    + "ST: %s\r\n" % (st_val, ) + "\r\n")
    sock = socket.socket(socket.AF_INET, socket.SOCK_DGRAM, socket.IPPROTO_UDP)
    sock.setsockopt(socket.IPPROTO_IP, socket.IP_MULTICAST_TTL, 5)
    sock.bind(('', 19011))

    sock.settimeout(timeout * 1.1)
    sock.sendto(ssdpRequest.encode(), (SSDP_ADDR, SSDP_PORT))
    ip_set = set()
    while True:
        try:
            data, addr = sock.recvfrom(10240)
            print(data.decode())
        except socket.timeout:
            break
        ip_set.add(addr[0])
    sock.close()
    return list(ip_set)

\end{verbatim}

