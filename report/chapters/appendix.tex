\chapter*{ПРИЛОЖЕНИЕ А} \label{app:appendix-usage-example}

\begin{center}
    \bfseries Пример использования разработанной библиотеки
\end{center}

\begin{verbatim}
    
#include <EEManager.h>


struct my_datatype {
    uint8_t a = 1;
    char[10] = { 0 };
};

void setup() {
    Serial.begin(115200);

    // Инициализация библиотеки
    EEMemManager::init();
    // Получение разделов
    MemPart part1 = EEMemManager::GetMemPart("part1");
    MemPart part2 = EEMemManager::GetMemPart("part2");

    // Получение переменной
    // Вывод порядкового номера запуска
    bool created;
    int loads_num = 1;
    EEPROMVar loads_num_var = part1.getVar("loads_num", &loads_num, &created);
    Serial.print("Load #");
    Serial.println(loads_num);
    if (created)
        Serial.println("First run. Counter was created in EEPROM.");
    else
        Serial.println("Not first run. Counter was found in EEPROM.");

    // Обновление порядкового номера запуска
    loads_num++;
    loads_num_var.updateNow();

    // Получение переменной сложного типа
    my_datatype my_var;
    EEPROMVar next_load_num_var = part1.getVar("testVar", &my_var);

    // Получение переменной с таким же именем, но из другого раздела
    int x = 4;
    EEPROMVar x_var = part2.getVar("testVar", &x);
    Serial.print("x ");
    Serial.println(x);

    x++;
    x_var.update();

    while (!x_var.tick()){
        Serial.println("Waiting...");
        delay(1);
    }
    Serial.println("Variable has been updated");

    // Чтение нового значения в другую переменную (локальная переменная x, в любом случае изменилась).
    int y;
    EEPROMVar x_var = part2.getVar("testVar", &y);
    Serial.print("New value: ");
    Serial.println(y);
}

void loop() {
    

}
    
\end{verbatim}

