\chapter*{ЗАКЛЮЧЕНИЕ}

В ходе работы были изучены принципы работы сетей Интернета вещей, подхода онтологически-управляемых периферийных вычислений и функционирования платформы SciVi.
Были проанализированы существующие технологии и средства для управления энергонезависимой памятью микроконтроллеров и самоидентификации и поиска устройств в сетях IoT.
На основе проведённого анализа были спроектированы собственные решения, которые были реализованы в виде программных модулей языка C++.
Разработанные модули удовлетворяют поставленным в работе требованиям, успешно прошли процесс тестирования и уже частично интегрированы в платформу SciVi.   

Таким образом, поставленные в работе задачи были решены, а её главная цель - достигнута.
