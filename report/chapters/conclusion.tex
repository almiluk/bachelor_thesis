\chapter*{ЗАКЛЮЧЕНИЕ}

В ходе работы были изучены принципы работы сетей Интернета вещей, подхода онтологически управляемых периферийных вычислений и функционирования платформы SciVi. Были выявлены недостающие компоненты платформы, необходимые для комплексного решения на её основе задачи автоматизации программирования устройств Интернета вещей. Такими компонентами оказались средства автоматизации обнаружения периферийных устройств в локальной сети и средства эффективного взаимодействия с их энергонезависимой памятью.
С целью наращивания функциональности SciVi, были проанализированы существующие технологии и средства для управления энергонезависимой памятью микроконтроллеров, а также самоидентификации и поиска устройств в сетях IoT.
На основе проведённого анализа были спроектированы собственные решения, которые были реализованы в виде программных модулей языка C++.
Разработанные модули удовлетворяют поставленным в работе требованиям, успешно прошли процесс тестирования и частично интегрированы в платформу SciVi. 

Новая функциональность, добавленная в платформу SciVi в рамках этой работы, позволила вывести средства программирования устройств Интернета вещей в рамках концепции онтологически управляемых периферийных вычислений на новый уровень автоматизации.

Таким образом, поставленные в работе задачи были решены, а её главная цель - достигнута.
