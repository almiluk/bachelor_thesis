\chapter*{ЗАКЛЮЧЕНИЕ}

В ходе работы были изучены принципы работы сетей Интернета вещей, подхода онтологически управляемых периферийных вычислений и функционирования платформы SciVi. Были выявлены недостающие компоненты платформы, необходимые для комплексного решения на её основе задачи автоматизации программирования устройств Интернета вещей. Такими компонентами оказались средства автоматизации обнаружения периферийных устройств в локальной сети и средства эффективного взаимодействия с их энергонезависимой памятью.
С целью наращивания функциональности SciVi, были проанализированы существующие технологии и средства для управления энергонезависимой памятью микроконтроллеров, а также самоидентификации и поиска устройств в сетях IoT.
На основе проведённого анализа были спроектированы собственные решения, которые были реализованы в виде программных модулей языка \CPP.
Разработанные модули удовлетворяют поставленным в работе требованиям, успешно прошли процесс тестирования и частично интегрированы в платформу SciVi.

Новая функциональность, добавленная в платформу SciVi в рамках этой работы, позволила вывести средства программирования устройств Интернета вещей в рамках концепции онтологически управляемых периферийных вычислений на новый уровень автоматизации.

Таким образом, поставленные в работе задачи были решены, а её главная цель~--- достигнута.

Также была проведена апробация результатов данной работы:
\begin{enumerate}
	\item Доклад <<Разработка онтологически управляемого протокола сетевой самоидентификации устройств интернета вещей"' на VII студенческой научно-практической конференции "`Математическое и программное обеспечение информационных и интеллектуальных систем>> (ПГНИУ, 28-29 апреля 2022 г.).
	За выступление с докладом получена грамота 1 степени. Копия грамоты представлена в приложении \ref{appendix:certificate1}.
	\item Статья <<Разработка онтологически управляемого протокола сетевой самоидентификации устройств интернета вещей"' была опубликована в сборнике статей "`Актуальные проблемы математики, механики и информатики 2022>> \cite{incollection:odec-self-id-protocol}.
	\item Доклад <<Разработка средств автоматизации программирования устройств Интернета вещей на базе платформы SciVi"' на VIII студенческой научно-практической конференции "`Математическое и программное обеспечение информационных и интеллектуальных систем>> (ПГНИУ, 24 мая 2023 г.).
	За выступление с докладом получена грамота 3 степени. Копия грамоты представлена в приложении \ref{appendix:certificate2}.
\end{enumerate}
