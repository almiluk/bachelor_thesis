\chapter{Разработка библиотеки менеджера EEPROM}

\section{Уточнение требований к разрабатываемом библиотеке}

На основе требования об обеспечении независимости работы с EEPROM из различных программных модулей, с целью уменьшения количества коллизий пользовательских идентификаторов данных было принято решение реализовать следующее:
\begin{enumerate}
	\item Ввести иерархию идентификаторов блоков данных: не хранить все идентификаторы в едином пространстве имён, а создавать множество таких пространств и реализовать механизм обращения к ним по особым идентификаторам, которые должны быть уникальны между собой.
	\item Обязать разработчиков программных модулей, использующий разрабатываемую библиотеку, создавать для этих модулей отдельные пространства имён, идентификаторы в которых должны быть связаны с названием модуля, а непосредственно используемые блоки данных описывать внутри эти пространств имён.
	\item Обязать разработчиков самостоятельно поддерживать уникальность идентификаторов внутри отдельных моделей.
\end{enumerate}
Такой шаг позволит изолировать друг от друга идентификаторы, используемые различными разработчиками в различных программных модуля.
А вероятность коллизий идентификаторов пространств имён сама по себе не является большой.
Таким образом разработчики смогут использовать данную библиотеку, не задумываясь о других программных модулях, использующих её, что выполняет основное требование к разрабатываемой библиотеке.
В дальнейшем будем называть такие пространства имён разделами, а блоки данных, хранящиеся в EEPROM -- EEPROM-переменными.

На основе дополненного списка требований к библиотеке можно составить диаграмму прецедентов работы с данной библиотекой. Она представлена на рисунке \ref{fig:ucd}.

\myfigure[scale=0.75]{UCD}{Диаграммы прецедентов использования библиотеки}{fig:ucd}

На диаграмме отсутствует прецедент удаления переменной в связи с тем, что, как писалось выше, основное назначение применения EEPROM в программах для  микроконтроллеров -- сохранение состояния устройства между его запусками.
Как правило, для запуска микроконтроллера с одной и той же программой необходимы одни и те же данные, следовательно удалять их нет необходимости.

\section{Разработка структуры библиотеки}

\subsection{Общая структура библиотеки}

В соответствии с общепринятым стилем, библиотека должна быть написана с использованием объектно-ориентированной парадигмы программирования.
Основные понятия, которыми оперирует разрабатываемая библиотека -- раздел и EEPROM-переменная, следовательно, в библиотеке должны содержаться классы соответствующие. Кроме того необходим отдельный класс оперирующий EEPROM в целом, назовём его классом менеджера памяти.


\subsection{Внешний интерфейс библиотеки}

С точки зрения пользователя библиотеки, её классы должны выглядеть как показано на рисунке \ref{fig:inerface-class-diagram}. На данном рисунке EEPROMVar - класс, описывающий EEPROM-переменную, её публичные методы взяты из соответствующего класса библиотеки EEManager (пункт \ref{section:eemanager}). Его методы:
\begin{enumerate}
	\item updateNow - мгновенная запись нового значения в EEPROM.
	\item update - запланировать запись нового значения, то есть запустить таймер отложенной записи.
	\item tick - метод, который необходимо вызывать регулярно, если используется отложенная запись: реальная запить нового значения EEPROM произойдёт при первом вызове метода tick() после истечения задержки записи.
	\item getTimeout - вернуть текущее значение задержки записи.
	\item setTimeout - установить новое значение задержки записи.
\end{enumerate}

Класс MemPart описывает раздел EEPROM, его метод getVar возвращает EEPROM-переменную с именем равным значению параметра name и связывает её с локальной переменной произвольного типа T, указателем на которую является параметр data.

\mysvg[inkscapelatex=false,width=\columnwidth]{interface-cd}{Диаграмма классов библиотеки с точки зрения её пользователя}{fig:inerface-class-diagram}



\subsection{EEPROM-переменные}

\subsection{Разделы памяти}

\subsection{Менеджер памяти}

\section{Разработка библиотеки}

\section{Использование библиотеки}
