\chapter*{ВВЕДЕНИЕ}

Энергонезависимая память -- особый вид запоминающих устройств, способный хранить данные при отсутствии электропитания.
Такая память используется в составе вычислительных устройств, в том числе для хранения данных, необходимых для их инициализации, и конфигурационных данных между их запусками.

Задача хранения конфигурационных данных при отсутствии электропитания особо остро стоит при работе с микроконтроллерами.
Это обусловлено, во-первых, уязвимостью таких устройств к перебоям электропитания и, во-вторых, особенностями условий их использования: устройства с микроконтроллерами обычно создаются для автономной работы, поэтому после временного отключения питания они должны самостоятельно восстанавливать своё прошлое состояние.
В микроконтроллерах для решения этой задачи обычно используются электрически стираемые перепрограммируемые постоянные запоминающие устройства (ЭСППЗУ, англ. Electrically Erasable Programmable Read-Only Memory, EEPROM) -- вид устройств энергонезависимой памяти, позволяющих электрическим импульсом стереть сохранённые данные, а затем, при необходимости, записать новые \cite{incollection:eeprom-proposal, article:eeprom}.

Микроконтроллеры, в частности, используются платформой научной визуализации и визуальной аналитики SciVi, разработанной сотрудниками Пермского государственного национального исследовательского университета \cite{article:scivi, article:scivi-overview}.
В SciVi уже реализовано сохранение настроечной информации в EEPROM, однако сделано это за счёт стандартных средств. Их низкоуровневость и ограниченность не позволяют использовать EEPROM удобно и, главное, расширять его применение хранением новых данных.

В основе данной работы лежит поиск решения указанных проблем в применении стандартных средств использования EEPROM микроконтроллеров.

Цель работы: разработать программный модуль с высокоуровневым интерфейсом, позволяющий удобно и без необходимости ручной настройки сохранять и считывать информацию из EEPROM микроконтроллеров, в соответствии с требованиями платформы SciVi.

Объект исследования данной работы: автоматизация периферийных вычислений.

Предмет исследования: средства платформы SciVi для организации онтологически-управляемых периферийных вычислений.

Для достижения цели работы были поставлены следующие задачи:
\begin{enumerate}
	\item Составить требования к необходимому программному модулю.
	\item Исследовать существующие средства для работы с энергонезависимой памятью и, в частности, EEPROM микроконтроллеров.
	\item При возможности, выбрать одно из таких средств для использования в качестве основы разрабатываемого модуля.
	\item Разработать программный модуль для работы с EEPROM микроконтроллеров, соответствующий всем поставленным требованиям.
	\item Провести тестирование и отладку разработанного программного модуля.
	\item Интегрировать разработанный модуль в платформу SciVi.
\end{enumerate}
