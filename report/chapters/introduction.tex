\chapter*{ВВЕДЕНИЕ}

В настоящее время активно развивается концепция Интернета вещей (англ. Internet of Things, IoT)~--- сетей передачи данных между устройствами без непосредственного участия человека.
Интернет вещей находит применение в различных областях: от использования в сфере персональных устройств и систем <<умных домов>> до автоматизации производств и систем мониторинга.
В то же время, программирование устройств IoT остаётся достаточно сложным процессом в связи со множеством трудностей в организации сетевого взаимодействия и реконфигурации устройств.
Кроме того, этот процесс обычно требует значительной квалификации разработчика, что затрудняет использование IoT специалистами предметных областей конечных решаемых задач.

Со значительной частью обозначенных проблем помогает справиться применение подхода онтологически управляемых периферийных вычислений (англ. Ontology-Driven Edge Computing), который позволяет управлять устройствами IoT и их сетями только с помощью онтологий, без необходимости прибегать к классическому программированию \cite{incollection:odec}.
Использование данного подхода также позволяет создавать унифицированные интерфейсы устройств и объединять в общие сети разнородные вычислительные ресурсы \cite{incollection:eon-communications}.

Онтологически управляемые периферийные вычисления играют важную роль в работе платформы научной визуализации и визуальной аналитики SciVi \cite{article:scivi, article:scivi-overview}.
Платформа SciVi позволяет декларативно, с помощью графического редактора, описывать алгоритмы сбора, обработки и отображения данных, а также выполнять эти алгоритмы на различных устройствах, храня и передавая их в виде онтологий.
Благодаря этому, создаваемые алгоритмы могут быть исполнены в том числе на устройствах IoT, а сам процесс создания не требует особой квалификации программиста и может быть выполнен специалистом в области конкретной решаемой задачи.
Важными преимуществами платформы являются также расширяемость и адаптируемость: элементарные фрагменты, из которых составляются алгоритмы, также описываются онтологиями, и их список может быть легко пополнен операциями, необходимыми для решения конкретных пользовательских задач.

При этом, часть задач автоматизации процесса программирования устройств Интернета вещей на платформе SciVi ещё не решена. А именно, в SciVi отсутствуют средства автоматизации обнаружения периферийных устройств в локальной сети и средства эффективного взаимодействия с их энергонезависимой памятью.

Цель работы: создание комплексного решения по автоматизации программирования устройств Интернета вещей на базе платформы SciVi путём реализации недостающей функциональности в рамках концепции онтологически управляемых периферийных вычислений.

Объект исследования данной работы: автоматизация периферийных вычислений.

Предмет исследования: средства платформы SciVi для организации онтологически управляемых периферийных вычислений.

Для достижения цели работы были поставлены следующие задачи:
\begin{enumerate}
	\item Провести анализ литературы по тематике Интернета вещей и онтологически управляемых периферийных вычислений.
	\item Изучить принципы функционирования платформы визуальной аналитики SciVi.
	\item Провести анализ литературы и существующих решений в областях автоматизации обнаружения устройств Интернета вещей в локальной сети и взаимодействия с их энергонезависимой памятью.
	\item Спроектировать и разработать программные решения для автоматизации обнаружения устройств Интернета вещей в локальной сети и взаимодействия с их энергонезависимой памятью с учётом особенностей подхода онтологически управляемых периферийных вычислений и платформы SciVi.
	\item Интегрировать разработанные решения в платформу SciVi и провести комплексное тестирование средств автоматизации программирования устройств Интернета вещей на базе этой платформы.
\end{enumerate}
