\chapter{Анализ существующих решений}

\section{Постановка задачи}

В платформе SciVi в основном применяются микроконтроллеры серии ESP8266, а для их программирования используются инструменты среды разработки Arduino IDE, позволяющие программировать микроконтроллеры, используя язык программирования C++.
А так же специальное дополнение к этой среде для работы с ESP8266 \cite{web:esp-core}, содержащее, в частности, набор “стандартных” библиотек.

Таким образом, необходимый программный модуль должен представлять собой библиотеку классов языка программирования C++, может использовать стандартный набор библиотек Arduino IDE и указанного дополнения к ней, выполнятся на микроконтроллерах серии ESP8266 и, по возможности, Arduino.
Эта библиотека должна:
\begin{enumerate}
	\item Предоставлять пользователю возможность сохранять и считывать данные из EEPROM микроконтроллера. При этом:
	\begin{enumerate}
		\item Данные могут иметь произвольную структуру.
		\item Доступ к ним должен производиться по некоторым  идентификаторам, уникальным для различных данных. Без необходимости ручных манипуляций с адресами EEPROM со стороны пользователя.
	\end{enumerate}
	\item Автоматически определять факт наличия в EEPROM данных с заданным идентификатором, определять адрес для записи новых данных, сохранять в EEPROM метаданные о хранящих данных для их использования после перезапуска микроконтроллера.
	\item Минимизировать количество операций записи в EEPROM, т.к. каждая ячейка такой памяти может быть перезаписана ограниченное количество раз (обычно производители гарантируют от 100.000 до 1.000.000 циклов перезаписи), после чего выходит из строя.
\end{enumerate}

Ключевым требованием является полная автоматизация работы с адресами EEPROM, это необходимо для создания возможности использования EEPROM из различных независимых программных модулей.
В противном случае, этим модулям понадобилось бы каким-либо образом обмениваться информацией об используемых ими адресах для избежания чтения и записи разными модулями по одним и тем же адресам.


\section{Стандартная библиотека}

В стандартный набор библиотек Arduino IDE уже входит библиотека для работы с EEPROM \cite{web:arduino-eeprom}.
Однако она предоставляет только простые функции, такие как: записать и считать бит по указанному адресу, позже в неё были также добавлены функции для чтения и записи данных произвольных типов, но так же только по явно указанному адресу.
Очевидно, это делает стандартную библиотеку нарушающей все поставленные требования, однако её функции можно использовать в качестве низкоуровневого интерфейса EEPROM в разрабатываемом библиотеке.
Кроме указанных выше, стандартная библиотека содержит обёртку вокруг функции записи, производящую физическую перезапись данных, только если они отличаются от
хранящихся по указанному адресу в данный момент.
В дальнейшем, практически во всех случаях, есть смысл использовать для записи именно эту функцию с целью уменьшения износа EEPROM.


\section{Библиотека EEManager}

Как и SciVi в данный момент, большая часть проектов, хранящих какие-либо данные в EEPROM, ограничиваются использованием стандартной библиотеки.
И до недавнего времени в открытом доступе отсутствовали более высокоуровневые альтернативы.
Однако не так давно появилась новая библиотека, преследующая те же цели - EEManager \cite{web:eemanager}.
Она имеет открытый исходный код (опубликован под лицензией MIT \cite{web:MIT}) и документацию на русском языке.

Эта библиотека также требует ручного манипулирования с адресами, однако имеет следующие преимущества:
\begin{itemize}
	\item Реализован механизм отложенной записи: по-умолчанию данные записываются в EEPROM с заданной задержкой после последней команды на запись. Использование такого подхода имеет смысл в ситуациях, когда данные должны перезаписываться много раз за короткий промежуток времени, в действительности же, с таким механизмом, данные в EEPROM будут записаны только в последний раз. В то же время этот механизм имеет значительный недостаток: если потеря питания произойдёт после команды записи, но до истечения задержи, новые данные записаны не будут. Это делает использование такого механизма оправданным только в устройствах, для которых гарантия записи не является обязательной и точность восстановления состояния после потери питания не представляет критической важности.
	\item Библиотека также реализует "механизм ключа первой записи". Вместе с каждым блоком данных в EEPROM хранится специальное однобайтовый ключ. При обращении к блоку данных пользователь указывает придуманный им ключ, который не должен изменяться от запуска к запуску, а из EEPROM считывается записанное значение ключа. Если они совпадают, значит необходимые данные уже записаны в EEPROM и их необходимо считать, иначе данные никогда не были записаны, тогда данные должны быть наоборот записаны.
\end{itemize}

Работа с данной библиотекой осуществляется следующим образом:
\begin{enumerate}
	\item Создаётся переменная в энергозависимой памяти, значение которой необходимо хранить в EEPROM.
	\item Создаётся специальный объект, описывающий блок EEPROM. При этом пользователь указывает переменную в энергозависимой памяти, значение которой  необходимо хранить, и адрес в EEPROM, начиная с которого должна быть записана эта переменная.
	\item С помощью механизма ключа первой записи либо в EEPROM записывается значение переменной по-умолчанию, либо наоборот: сохранённое в EEPROM значение считывается в переменную.
	\item При необходимости текущее значение переменной записывается в EEPROM. Для этого у описанного выше объекта существует два метода: для немедленной записи и для запуска таймера записи с задержкой.
\end{enumerate}


\section{EEPROMWearLevel}
