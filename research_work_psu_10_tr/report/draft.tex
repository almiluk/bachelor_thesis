\documentclass{report}
\usepackage[T2A]{fontenc}

\usepackage[english, russian]{babel}
\title{Разработка средств автоматизации программирования устройств Интернета вещей на базе платформы SciVi}
\author{Лукьянов Александр Михалович}
\date{\the\year{}}

\begin{document}
\maketitle

\tableofcontents

\addcontentsline{toc}{chapter}{Введение}
\chapter*{Введение}

\paragraph{} Энергонезависимая память --- особый вид запоминающих устройств, способный хранить данные при отсутствии электропитания. Такая память чаще всего используется для хранения конфигурационных данных и данных, необходимых для инициализации устройства, между запусками устройства. Такая задача особо остро стоит при работе с микроконтроллерами, так как они, во-первых, уязвимы к проблемам с перебоями в электропитании и, во-вторых, устройства с ними обычно создаются для автономной работы, поэтому после временного отключения питания они должны самостоятельно восстанавливать своё прошлое состояние. В микроконтроллерах для решения этой задачи обычно используются электрически стираемые перепрограммируемые постоянные запоминающие устройства (ЭСППЗУ, англ. Electrically Erasable Programmable Read-Only Memory, EEPROM) --- вид устройств энергонезависимой памяти, позволяющий электрическим импульсом стереть сохранённые данные, а затем записать новые.

\paragraph{} Микроконтроллеры, в частности, используются в платформе научной визуализации и визуальной аналитики SciVi, разработанная сотрудниками Пермского государственного национального исследовательского университета \cite{SciVi-main}.

\chapter{Анализ существующих решений}

\section{Постановка задачи}

\section{Стандартная библиотека}

\section{Библиотека EEManager}


\chapter{Разработка библиотеки менеджера EEPROM}

\section{Требования к библиотеки}

\section{Выбор необходимых программных средств}

\section{Разработка структуры библиотеки}

\subsection{Разработка внешнего интерфейса библиотеки}

\subsection{Переменные}

\subsection{Разделы памяти}

\subsection{Менеджер памяти}

\section{Разработка библиотеки}

\section{Использование библиотеки}


\addcontentsline{toc}{chapter}{Заключение}
\chapter*{Заключение}

\addcontentsline{toc}{chapter}{Библиографический список}
\bibliographystyle{plain}
\bibliography{sources}

\end{document}